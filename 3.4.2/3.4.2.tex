\documentclass[14pt]{article}

\usepackage[utf8x]{inputenc}
\usepackage[russian]{babel}
\usepackage{graphicx}
\graphicspath{{images/}}
\DeclareGraphicsExtensions{.pdf,.png,.jpg}

\usepackage{amsmath}
\usepackage{pgfplots}

\usepackage{geometry} % Меняем поля страницы
\geometry{left=2cm}% левое поле
\geometry{right=1.5cm}% правое поле
\geometry{top=2cm}% верхнее поле
\geometry{bottom=2cm}% нижнее поле

\renewcommand{\theenumi}{\arabic{enumi}}
\renewcommand{\labelenumi}{\arabic{enumi}}
\renewcommand{\theenumii}{.\arabic{enumii}}
\renewcommand{\labelenumii}{\arabic{enumi}.\arabic{enumii}.}
\renewcommand{\theenumiii}{.\arabic{enumiii}}
\renewcommand{\labelenumiii}{\arabic{enumi}.\arabic{enumii}.\arabic{enumiii}.}

\begin{document}
\begin{titlepage}
	\begin{center}
		\fontsize{18pt}{20pt}\selectfont
		\textbf{Работа 3.3.5.}	
	
		\vspace{5cm}
		\fontsize{24pt}{25pt}\selectfont
		Закон Кюри-Вейсса
	\end{center}
	\begin{flushright}
		\fontsize{18pt}{20pt}\selectfont
		\vspace{14cm}
		\hspace{-3cm}
		\textit{Корнеев Е.С.}
	\end{flushright}		
\end{titlepage}

\begin{center}
	\fontsize{16pt}{18pt}\selectfont	
	Закон Кюри-Вейсса
\end{center}


\fontsize{14pt}{16pt}\selectfont
\vspace{1cm}
\textbf{Цель работы:} изучение температцрной зависимости магнитной восприимчивости ферромагнетика выше точки Кюри.

\vspace{0.5cm}
\textbf{Оборудование:} катушка самоиндукции с образцом из гадолиния, термостат, частотомер, цифровой вольтметр, LC-автогенератор, термопара медь-константан. 

\vspace{1cm}
Вещества с атомными магнитными моментами, отличными от нуля, обладают парамагнитными свойствами. Внешнее магнитное поле ориентирует магнитные моменты, которые в отсутствии его располагались в хаотичном порядке. При повышении температуры парамагнетика дезориентирующее воздействие теплового движения возрастает, и магнитная восприимчивость спадает (в постоянном магнитном поле) по закону Кюри:
$$
	\chi = \frac{C}{T},
$$
\noindent где $C$ - постоянная Кюри.

При $T \rightarrow 0$ тепловое движение все меньше препятствует магнитным моментам атомовориентироваться в одном направлении при действии сколь угодно малого внешнего поля, в связи с чем $\chi$ неограниченно возрастает. В ферромагнетиках это происходит при понижении температуры не до 0, а до температуры Кюри $\Theta$. Оказывается, для ферромагнетиков закон Кюри должен быть заменен заоном Кюри-Вейсса:
$$
	\chi \sim \frac{1}{T - \Theta_p},
$$
\noindent где $\Theta_p$ - температура, близкая к $\Theta$.

В данной работе изучается зависимость $\chi(T)$ гадолиния, так как его точка Кюри лежит в интервале комнатных температур.

\vspace{1cm}
\textbf{Экспериментальная установка:} схема приведена на рисунке. Рабочий образец из гадолиния расположен внутри пустотелой катушки самоиндукции, входящей в состав колебательного контура, входящего в состав LC-автогенератора. Гадолиний - хороший проводник электрического тока, а рабочая частота генератора достаточно велика, поэтому для уменьшения вихревых токов образец изготовлен из мелких кусочков размером около 0.5 мм. Катушка 1 с образцом помещена в стеклянный сосуд 2, залитый трансформаторным маслом, предохраняющим образец от окисления и способствующим ухудшению контакта между кусочками образца, а также улучшающим тепловой контакт между образцом и жидкостью 3 в термостате. Термометр 4 используется для определения температуры термостата. 

Магнитную восприимчивость определяем по изменению самоиндукции катушки. Обозначив через $L$ самоиндукцию катушки с образцом, $L_0$ - без образца, получим 
$$
	(L - L_0) \sim \chi
$$
При изменении самоиндукции образа изменяется период колебаний автогенератора:
$$
	\tau = 2\pi\sqrt{LC},
$$
\noindent где $C$ - емкость контура. Период колебаний в отсутствие образца определяется самоиндукцией пустой катушки:
$$
	\tau_0 = 2\pi\sqrt{L_0C}
$$

Отсюда получим:
$$
	(L - L_0) \sim (\tau^2 - \tau_0^2)
$$
\noindent или
$$
	\chi \sim (\tau^2 - \tau_0^2)
$$

Отсюда видно:
$$
	(T - \Theta_p) \sim \frac{1}{(\tau^2 - \tau_0^2)}
$$




\end{document}