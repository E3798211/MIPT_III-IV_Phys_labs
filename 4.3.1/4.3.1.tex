\documentclass[14pt]{article}

\usepackage[utf8x]{inputenc}
\usepackage[russian]{babel}
\usepackage{graphicx}
\graphicspath{{images/}}
\DeclareGraphicsExtensions{.pdf,.png,.jpg}

\usepackage{amsmath}
\usepackage{pgfplots}

\usepackage{geometry} % Меняем поля страницы
\geometry{left=2cm}% левое поле
\geometry{right=1.5cm}% правое поле
\geometry{top=2cm}% верхнее поле
\geometry{bottom=2cm}% нижнее поле

\renewcommand{\theenumi}{\arabic{enumi}}
\renewcommand{\labelenumi}{\arabic{enumi}}
\renewcommand{\theenumii}{.\arabic{enumii}}
\renewcommand{\labelenumii}{\arabic{enumi}.\arabic{enumii}.}
\renewcommand{\theenumiii}{.\arabic{enumiii}}
\renewcommand{\labelenumiii}{\arabic{enumi}.\arabic{enumii}.\arabic{enumiii}.}

\begin{document}
\begin{titlepage}
	\begin{center}
		\fontsize{18pt}{20pt}\selectfont
		\textbf{Работа 4.3.1.}	
	
		\vspace{5cm}
		\fontsize{24pt}{25pt}\selectfont
		Изучение дифракции света
	\end{center}
	\begin{flushright}
		\fontsize{18pt}{20pt}\selectfont
		\vspace{14cm}
		\hspace{-3cm}
		\textit{Корнеев Е.С.}
	\end{flushright}		
\end{titlepage}

\begin{center}
	\fontsize{16pt}{18pt}\selectfont	
	Изучение дифракции света
\end{center}


\fontsize{14pt}{16pt}\selectfont
\vspace{1cm}
\textbf{Цель работы:} изучение дифракции Френеля и Фраунгофера.

\vspace{0.5cm}
\textbf{Оборудование:} оптическая скамья, ртутная лампа, монохроматор, щели с регулируемой шириной, рамка с вертикальной нитью, двойная щель, микроскоп на поперечных салазках с микрометрическим винтом, зрительная труба.

\vspace{1cm}
\textbf{Дифракция Френеля}

Схема установки для наблюдения дифракции Френеля представлена на рис. 1. Световые лучи освещают щель $S2$ и испытывают на ней дифракцию. Дифракционная картина рассматривается с помощью микроскопа $M$, сфокусированного на некоторую плоскость наблюдения $\Pi$.


Щель $S2$ освещается параллельным пучком монохроматического света с помощью коллиматора, образованного объективом $O1$ и щелью $S1$, находящейся в его фокусе. На щель $S1$ сфокусировано изображение спектральной линии, выделенной из спектра ртутной лампы $Л$ при помощи простого монохроматора $C$, в котором используется призма прямого зрения. Распределение интенсивности света в плоскости наблюдения $\Pi$ проще всего рассчитывать с помощью зон Френеля (для щели их иногда называют зонами Шустера). При освещении щели $S2$ параллельным пучком лучей (плоская волна) зоны Френеля представляют собой полоски, параллельные краям щели (рис. 2). Результирующая амплитуда в точке наблюдения определяется суперпозицией колебаний от тех зон Френеля, которые не перекрыты створками щели. Графическое определение результирующей амплитуды производится с помощью векторной диаграммы -- спирали Корню. Суммарная ширина $m$ зон Френеля $z_m$ определяется соотношением
$$
	z_m = \sqrt{am\lambda}
$$

где
$a$ -- расстояние от щели до плоскости наблюдения (рис. 1), а $\lambda$ -- длина волны.

Вид наблюдаемой дифракционной картины определяется \textsl{числом Френеля} $\Phi$: квадрат числа Френеля
$$
	\Phi = \frac{D}{\sqrt{a\lambda}} ~-
$$

это отношение ширины щели $D$ к размеру первой зоны Френеля, т.е. число зон Френеля, которые укладываются на ширине щели. Обратную величину называют \textsl{волновым параметром}
$$
	p = \frac{1}{\Phi} = \frac{\sqrt{a\lambda}}{D}
$$

Дифракционная картина отсутствует, когда плоскость наблюдения $\Pi$ совпадает с плоскостью щели: при $\Phi \rightarrow \infty$ мы имеем дело с геометрической оптикой. При небольшом удалении от щели, когда число Френеля $\Phi \gg 1$ (на щели укладывается огромное число зон), распределение интенсивности света за щелью также можно получить с помощью законов геометрической оптики (приближённо). Дифракционная картина в этом случае наблюдается только в узкой области на границе света и тени у краёв экрана.

При последующем небольшом удалении от щели (или изменении ширины щели $S2$) эти две группы дифракционных полос перемещаются практически независимо друг от друга. Каждая из этих групп образует картину дифракции Френеля на краю экрана. Распределение интенсивности при дифракции света на краю экрана может быть найдено с помощью спирали Корню. 

При дальнейшем увеличении расстояния a (или уменьшении ширины щели $S2$) обе системы дифракционных полос постепенно сближаются и, наконец, при $\Phi \geq 1$ накладываются друг на друга. Распределение интенсивности в плоскости наблюдения в этом случае определяется числом зон Френеля, укладывающихся на полуширине щели. Если это число равно $m$, то в поле зрения наблюдается $n = m - 1$ тёмных полос. Таким образом, по виду дифракционной картины можно оценить число зон Френеля на полуширине щели.


















\end{document}